\documentclass{article}

\usepackage[utf8]{inputenc} 
\usepackage[spanish]{babel} 
\usepackage{mathrsfs}
\usepackage[a4paper, left=1.5cm, right=1.5cm, top=1.5cm, bottom=1.5cm]{geometry} 
\usepackage{amsmath, amsthm, amssymb, latexsym}
\usepackage{amsfonts}
\usepackage{booktabs}
\usepackage[dvipsnames,table,xcdraw]{xcolor}
\usepackage[none]{hyphenat}
\usepackage{graphicx}
\usepackage{sectsty}
\usepackage{fancyhdr}
\usepackage{transparent}
\usepackage{cancel}
\usepackage[normalem]{ulem}
%COMANDOS--------------------------------------------------------------------------------------------------------------------------------------------
\newcommand{\ig}[2]{\begin{minipage}[c]{#1cm}
    \includegraphics[width=#1cm]{imgs/#2.jpeg}
\end{minipage}}

\newcommand{\xo}[1]{\paragraph*{#1}}

\newcommand{\tb}[1]{\textbf{\textcolor{blue}{#1}}}

\newcommand{\f}[1]{$$#1$$}

\newcommand{\pp}[1]{\left(#1\right)}

\newcommand{\q}[2]{\frac{#1}{#2}}

\newcommand{\nt}[1]{
\hrule
\texttransparent{0.8}{\footnotesize{
\xo{Notas:}#1
}}
\newpage}

\newcommand{\mb}[1]{\mathbb{#1}}

\newcommand{\N}[0]{\mb{N}}

\newcommand{\R}[0]{\mb{R}}


\newcommand{\al}[1]{\begin{align*}
#1
\end{align*}}

\newcommand{\s}[2]{#1   &   #2\\}

\newcommand{\ee}[0]{\varepsilon}
%COMANDOS--------------------------------------------------------------------------------------------------------------------------------------------

%DISEÑO----------------------------------------------------------------------------------------------------------------------------------------------

\fancyhf{}
\lhead[\leftmark]{Andy F. Logacho}
\rhead[Andy F. Logacho]{\rightmark}
\lfoot[\thepage]{Cálculo Integral}
\rfoot[Cálculo Integral]{\thepage}
\renewcommand{\headrulewidth}{0.5pt}
\renewcommand{\footrulewidth}{0.5pt}

\pagestyle{fancy}
%DISEÑO----------------------------------------------------------------------------------------------------------------------------------------------

\begin{document}
\begin{titlepage}
    \centering
    {\includegraphics[width=0.2\textwidth]{Logo_EPN.eps}\par}
    \vspace{1cm}
    {\bfseries\LARGE Escuela Politécnica Nacional \par}
    \vspace{1cm}
    {\scshape\Large Facultad de Ciencias \par}
    \vspace{3cm}
    {\scshape\Huge Calculo Integral \par}
    {\itshape\Large Apuntes de Clase \par}
    \vfill
    {\itshape\Large Andy F. Logacho \par}
    \vfill
    {\Large Profesor:  \par}
    {\Large Herrera Teran Maribel Kateryn\par}
    \vfill
    {\Large Mayo 2023 \par}
\end{titlepage}
\newpage

\section*{Motivación}
Utilice rectángulos para estimar el área bajo la parábola $y=x^2$, desde 0 hasta 1.

\begin{minipage}[c]{5cm}
    \includegraphics[width=5cm]{imgs/1.jpeg}
\end{minipage}
\ig{12}{2}

\xo{Solución:}El área de la función $f(x)=x^2$ viene a estar entre 0 y 1, dado que se encuentra dentro de un cuadrado de área 1. 
Se puede deducir un mejor análisis si dividimos a $S$ en 4 franjas $(S_1,S_2,S_3,S_4)$, donde las verticales se trazan cuando 
$x=\frac{1}{4},x=\frac{1}{2},x=\frac{3}{4},x=1$ siendo esta altura los supremos de los intervalos. 

Ahora podemos obtener un área aproximada formando rectángulos donde su altura son las imágenes de $x$ antes mencionadas y el ancho de todos 
los rectángulos son $\frac{1}{4}$. Si denotamos a $R_4$ como la suma de el área de todos los rectángulos (Área por la derecha) tenemos:

\f{R_4=\frac{1}{4}\pp{\frac{1}{4}}^2+\frac{1}{4}\pp{\frac{2}{4}}^2+\frac{1}{4}\pp{\frac{3}{4}}^2+\frac{1}{4}(1)^2=\q{15}{32}=0.46875}

Lo que se deduce que el área de $S$ es menor a $R_4$ (decimos que es menor dado que el área de los rectángulos sobresalen de $S$)

\ig{5}{3} \tb{Ahora utilizaremos rectángulos mas pequeños.} 

Se continua con la misma idea con la diferencia de que los rectángulos son mas pequeños, 
donde las imágenes de $x$ cuando $x=0,x=\q{1}{2},x=\q{2}{4},x=\q{3}{4}$ son la altura, la cual a su vez es el ínfimo de los intervalos. Concluyendo definimos a 
$L_4$ como la suma de el área de estos rectángulos (Área por la izquierda), teniendo asi:

\f{L_4=\q{1}{4}(0)^2+\q{1}{4}\pp{\q{1}{4}}^2+\q{1}{4}\pp{\q{2}{4}}^2+\q{1}{4}\pp{\q{3}{4}}^2=\q{7}{32}=0.21875}

Podemos notar que ahora el área aproximado de $S$ es mayor a 0.21875, dado que los rectángulos no abarcan todo el área $S$, por lo que se concluye que:
\f{0.21875 < S < 0.46875}

\f{\quad}

\f{\quad}

\f{\quad}

\f{\quad}

\f{\quad}



\nt{Se presenta la idea esencial de la suma de Riemann y la integral definida, su historia y conceptos necesarios para su definición formal. 
Se mencionan conceptos como Ínfimo, Supremo, particiones y etiquetas. También se explica que $R_n$ representa el área por la derecha utilizando como altura las imágenes de los 
supremos de cada intervalo y $L_n$ representa el área por la izquierda utilizando como altura las imágenes de los ínfimos de los intervalos.
}



Dado que estos resultados son aproximaciones muy ambiguas de el área total de $S$, para la búsqueda de algo mas preciso tenemos que partir a $S$ en 
secciones más pequeñas, así al partirlo en 10, 20, 30, 50, 100 y 1000, obtenemos las siguientes áreas:
\begin{center}
    \ig{5}{4}    
\end{center}
Se puede notar que el área por derecha $(R_n)$ y por izquierda $(L_n)$ tienden a 0.33... cuando el valor de $n$ es mucho mayor.

Ahora, para obtener el área de forma precisa necesitamos la mayor cantidad de particiones (rectángulos) para obtener un área más precisa, evaluamos esto en 
$n-particiones$ con $n\in \N$, ási podemos definir a $R_n$, obteniendo:

\al{
\s{R_n}{=\q{1}{n}\pp{\q{1}{n}}^2+\q{1}{n}\pp{\q{2}{n}}^2+\q{1}{n}\pp{\q{3}{n}}^2+\cdots + \q{1}{n}\pp{\q{n}{n}}^2}
\s{}{=\q{1}{n^3}\pp{1^2+2^2+3^2+...+n^2}}
\s{}{=\q{1}{n^3}\pp{\q{n(n+1)(2n+1)}{6}}}
\s{}{=\q{(n+1)(2n+1)}{6n^2}}
}
Aplicamos límites para hallar el valor preciso de $S$, recordando que $S=R_n$
\al{
\s{\lim_{n\to{\infty}}{\dfrac{\left(n+1\right)\,\left(2\,n+1\right)}{6\,n^{2}}}}{=\dfrac{1}{6}\,\lim_{n\to{\infty}}{\dfrac{\dfrac{3}{n}+\dfrac{1}{n^{2}}+2}{1}}}
\s{}{=\dfrac{1}{6}\,\lim_{n\to{\infty}}{\dfrac{{{\dfrac{3}{n}}^{\,{{}}}}+{{\dfrac{1}{n^{2}}}^{\,{{}}}}+2}{1}}}
\s{}{=\dfrac{1}{6}\,\lim_{n\to{\infty}}{\dfrac{2}{1}}}
\s{}{=\q{1}{3}}
}

Asi se determina que el área $S$ es $\q{1}{3}$, además se puede reconocer que 
\f{S=\lim_{n\to \infty}R_n=\lim_{n\to \infty}L_n=\q{1}{3}}
Generalizando, se plantea el problema del área y se establece el buscarla mediante rectángulos, obteniendo un aproximación muy ambigua, por lo que se deduce la necesidad 
de generar un mayor numero de particiones, de tal modo que tienda al infinito, así introducimos los límites que ayudarán a conseguir de forma precisa el área. La formula general que 
se usó en todo este proceso fue:

\f{S=\lim_{x\to \infty}\sum_{i=1}^{n}f(x_i)\cdot \Delta x}
Donde $S$ es el área buscada, $f(x_i)$ es la altura de los rectángulos y $\Delta x$ es el grosor de estos.
\f{\quad}
\f{\quad}
\f{\quad}


\nt{Se concluye con la ircorporación de la límites para la obtención de el área buscada de forma precisa, manteniendo los conceptos del caso anterior y finalizando con 
la prévia de la formula que se usará en Suma de Riemann.}

\begin{center}\texttransparent{0.5}{\textbf{\Huge{
    1\\
    Cálculo de Primitivas de una Función Real de Variable Real}\\
    $\quad$
    }}
    \hrule
\end{center}
\setcounter{section}{1}
\subsection{Suma de Riemann}
Se considera $a,b\in \R$ con $a<b$

\xo{Definición 1:}Dado un intervalo $[a,b]$, una \tb{Partición del intervalo} $[a,b]$ de orden $n$ es un conjunto
\f{P=\{x_k \in \R : k=0,...,n\}}
tal que 
\f{a=x_0<x_1<...<x_n=b}

\begin{itemize}
    \item A los conjuntos 
    \f{I_k=[x_{k-1},x_k]}
    con $k=1,...n$ se los llama \tb{Subintervalos de la partición}

    \item A las cantidades 
    \f{\Delta x_k = x_k - x_{k-1}}
    con $k=1,...,n$ se la llama \tb{Longuitud} del subintervalo $I_k$

    \item A un conjunto 
    \f{C=\{c_k\in I_k:k=1,...,n\}}
    Se lo llama \tb{conjunto de etiquetas} para $P$

    \item A un par $(P,C)$ se lo llama una \tb{partición etiquetada} de $[a,b]$
    
    \item A la cantidad 
    \f{|P|=\max_{k=1,..,n}\Delta x_k}
    se lo llama \tb{grosor} de la partición
\end{itemize}

\xo{Definición 2:} Sean $f:[a,b]\to \R$ una función y $(P,C)$ una partición etiquetada de $[a,b]$ de orden $n$, entonces:
\f{S(f,P,C)=\sum_{k=1}^n f(c_k) \Delta x_k}
es la \tb{Suma de Riemann de $f$ respecto a la partición etiquetada $(P,C)$}

\xo{Definición 3:} Sean $f:[a,b]\to \R$ una función e $I\in \R$, se dice que el número $I$ es la \tb{Integral de $f$ sobre $[a,b]$}, denotado por:

\f{I=\int_a^bf(x)dx}
si $\forall \ee >0$, $\exists \delta >0$ tal que para toda partición $P$ que cumple con $|P|<\delta $ y todo conjunto de etiquetas $C$ de $P$, se tiene que 

\f{|S(f,P,C)-I| \leq \ee}

De existir el número $I$, se dice que $f$ es \tb{Integrable según Riemann}.

\nt{
Partición del intervalo es la cantidad de veces que se fraccionará el intervalo $[a,b]$.\\
Un sub intervalo de partción no es mas que el intervalo que va del ínfimo al supremo de cada frácción. En la Motivación, estos intervalos eran: $[0, \q{1}{4}],[\q{1}{4},\q{1}{2}],[\q{3}{4},1]$\\
La longitud del subintervalo no es mas que la diferencia entre el supremo y el ínfimo de cada subintervalo.\\
El conjunto de etiquetas son, los valores de $x$ que se evaluarán en la función para obtener la altura de los rectángulos, pueden sera el ínfimo, supremo o cualquier valor dentro del subintervalo.\\
El grosor se refiere de igual forma a la longitud, donde el la diferencia entre $c_k - c_{k-1}$ tiene que tener obligatoriamente el mismo valor que la longitud.
}

\subsection{Notación Sigma}
\xo{Definición 4:} Sean $n\in \N$ y $a_0,a_1,...,a_n\in \R$. La suma desde $k$ igual a $0$ hasta $n$ de $a_k$, notado por:
\f{\sum_{k=0}^n a_k}
se define de manera recursiva de la siguiente forma:
\begin{itemize}
    \item $$\sum_{k=0}^0=a_0$$
    
    \item $$\sum_{k=0}^{n+1}a_k=\sum_{k=0}^{n}a_k+a_{n+1}$$
\end{itemize}

Aquí es importante recalcar que no existe una única forma de escribir la notación Sigma.

\xo{Teorema 1:}Sean $n\in N$, $a_0,a_1,...,a_n\in \R$, $b_0,b_1,...,b_n\in \R$ y $c\in \R$. Se tiene que:

\begin{itemize}
    \item \f{\sum_{k=0}^{n}(a_k+b_k)=\sum_{k=0}^{n}a_k+\sum_{k=0}^{n}b_k}
    \item \f{\sum_{k=1}^{n}1=n}
    \item \f{\sum_{k=0}^{n}c\cdot a_k= c\cdot \sum_{k=0}^{n}a_k}
    \item \f{\sum_{k=0}^{n}c=(n+1)\cdot c}
    \item \f{\sum_{k=1}^{n}c=n\cdot c}
    \item \f{\sum_{k=1}^{n}k=\q{n(n+1)}{2}}
    \item \f{\sum_{k=1}^{n}k^2=\q{n(n+1)(2n+1)}{6}}
\end{itemize}

\f{\quad}
\f{\quad}
\f{\quad}
\f{\quad}
\f{\quad}
\f{\quad}
\f{\quad}
\f{\quad}
\f{\quad}
\nt{La herramienta de la notación sigma se utilizará principalmente para establecer de forma mas simplificada 
la suma de Riemann, por lo que se presentan algunas propiedades de la sumatoria y se reconoce la no existencia de 
una unica forma de escribir esta.}

\end{document}